
% !TEX TS-program = pdflatex
% !TEX encoding = UTF-8 Unicode

% This is a simple template for a LaTeX document using the "article" class.
% See "book", "report", "letter" for other types of document.

\documentclass[11pt]{article} % use larger type; default would be 10pt

\usepackage[utf8]{inputenc} % set input encoding (not needed with XeLaTeX)

%%% Examples of Article customizations
% These packages are optional, depending whether you want the features they provide.
% See the LaTeX Companion or other references for full information.

%%% PAGE DIMENSIONS
\usepackage{geometry} % to change the page dimensions
\geometry{a4paper} % or letterpaper (US) or a5paper or....
% \geometry{margin=2in} % for example, change the margins to 2 inches all round
% \geometry{landscape} % set up the page for landscape
%   read geometry.pdf for detailed page layout information

\usepackage{graphicx} % support the \includegraphics command and options

% \usepackage[parfill]{parskip} % Activate to begin paragraphs with an empty line rather than an indent

%%% PACKAGES
\usepackage{booktabs} % for much better looking tables
\usepackage{array} % for better arrays (eg matrices) in maths
\usepackage{paralist} % very flexible & customisable lists (eg. enumerate/itemize, etc.)
\usepackage{verbatim} % adds environment for commenting out blocks of text & for better verbatim
\usepackage{subfig} % make it possible to include more than one captioned figure/table in a single float
% These packages are all incorporated in the memoir class to one degree or another...

%%% HEADERS & FOOTERS
\usepackage{fancyhdr} % This should be set AFTER setting up the page geometry
\pagestyle{fancy} % options: empty , plain , fancy
\renewcommand{\headrulewidth}{0pt} % customise the layout...
\lhead{}\chead{}\rhead{}
\lfoot{}\cfoot{\thepage}\rfoot{}

%%% SECTION TITLE APPEARANCE
\usepackage{sectsty}
\allsectionsfont{\sffamily\mdseries\upshape} % (See the fntguide.pdf for font help)
% (This matches ConTeXt defaults)

%%% ToC (table of contents) APPEARANCE
\usepackage[nottoc,notlof,notlot]{tocbibind} % Put the bibliography in the ToC
\usepackage[titles,subfigure]{tocloft} % Alter the style of the Table of Contents
\renewcommand{\cftsecfont}{\rmfamily\mdseries\upshape}
\renewcommand{\cftsecpagefont}{\rmfamily\mdseries\upshape} % No bold!

\makeatletter
   \newcommand\figcaption{\def\@captype{figure}\caption}
   \newcommand\tabcaption{\def\@captype{table}\caption}
\makeatother
%%% END Article customizations

%%% The "real" document content comes below...

\title{CSE 509 Lecture 15}

\author{Prof. Don Porter, Scribe:Arun Rathakrishnan}
%\date{} % Activate to display a given date or no date (if empty),
         % otherwise the current date is printed 

\begin{document}
\maketitle
\section {Introduction}
Early browsers had to render only html pages and security was not a major goal
for browser developers. Modern browsers support javascript, plugins, CSS,
and web services are more powerful. That Chrome OS exposes the OS API through browser
explains the evolution of browsers. Websites co-exist in a common browsing environment
and browser's responsibility is to protect one site from accessing another site's
data as a way of ensuring confidentiality. Browsers must also protect themselves
and other web pages displayed from being compromised when displaying a malicious
website.

\section {Current Secuirty Trends}
Browsers like Google Chrome, allows each tab (browsing instance) to execute in
its own address space as a separate process. As a result, a malicious web page
in a tab, can not bring down other processes. But with the prevalence of mashups
where a part of browser estate(in a iframe), can be delegated to a different
web site (like the ads area in a page). For example, a google calendar widget
may be embedded in a travel planer website. In Chrome, both the entities will
share the same process. It is important to ensure that one entity does not
access or control the other, for the sake of security.  Also, browsers entrust the security of plugin contents to native applications like JVM, Flash Content and PDF readers. This is known to cause several security vulnerabilities.
\section {Same Origin Policy}
\subsection {Origin}
A resource is identified by URL or URI. The origin of a resource with respect to
Same Origin Policy is represented by the triple \{Protocol, Domain, Port\}. In
a nutshell two resources having the same origin can have access to each other,
while all other accesses between resources in general are disallowed.

\subsection {Examples}
\begin {table}[htp]
{\small
 \tabcolsep=0.11cm
    \begin{tabular}{ | l | l | l |}
	\hline
URL 1 & URL 2 & Same/Different Origin \\ \hline
http://www.eg.com/dir1/pg2 & http://www.eg.com/dir2/pg1 & Same Origin \\ \hline
http://www.secure.eg.com/pg & http://www.eg.com/pg & Different Origins (Hostname differs) \\ \hline
http://www.user@eg.com/pg & http://www.eg.com/pg & Same Origin (Hostname is still the same) \\ \hline
http://eg.com & http://www.eg.com & Different Origins (Hostname is different) \\ \hline
http://www.eg.com & http://v2.www.eg.com & Different Origins (Hostname is different) \\ \hline
https://www.eg.com & http://www.eg.com & Different Origins (Protocol is different) \\ \hline
http://www.eg.com:80 & http://www.eg.com:81 & Different Origins (Port is different) \\ \hline
\end {tabular}
}
\end {table}
\subsection {DOM}
Document Object Model provides an API for accessing the browser's internal
representation of the web page being displayed. It allows scripts to modify the
DOM elements in response to events. Same Origin Policy dictates that a DOM element
can be accessed by only a script with the same origin. If not for the policy
script from a malicious site can steal data by looking at the DOM of another site.
Usually the SOP is implemented by the browser from the origin information of
HTML elements and scripts.\\

The HTML elements in DOM have a heirarchical relationship - an enclosing tag is
a parent of all inner tags. Thus the entire document can be modelled as a tree.
A mashup page has several elements with different origins in the same tree, where
elements with disparate origins must not access each other.

\subsection {Cookies}
Cookies are used by webservers to track users. Since HTTP is a stateless protocol
browser resident cookies play an important role in tracking a user session. A
variant of SOP model governs Cookie acces, where origin is identified by path and
host name. For example cookies set by http://www.example.com/service can be accessed
by any descendant of the path like http://www.example.com/service/first/test
while any non-descendant like http://www.example.com/home.html can not access the
cookie set by the service URL. Similarly marking a cookie as secure means that
only a secure protocol can access its value.

\subsection {Exceptions}
Some times it makes sense for two different origins resources from the same 
super-domain to have access to each other. For example, it's reasonable for
mail.google.com to access DOM element of ads.google.com and viceversa. The
web developers circumvent SOP by setting the same document.domain value for
the two pages. In the case of domains hosted as a part of hosting service,
elements hacker.domain.com should not accessed by attacker.domain.com.\\

Javascript source imported through the script tag from different origins,
assume the origin of the importer. So a script imported from google.com
by mypage.com can modify DOM elements of mypage.com as it has the origin
reset to mypage.com. But as result, the script can not access elements from 
google.com.

\subsection {Cross-site Scripting Attack}
Cross site scripting attempts to bypass SOP policy, to enable a script from a
different domain access DOM elements of a page. For example, in a Youtube comment,
a user can leave a malicious script as a comment, that may be wrongly considered
a plain text comment. Now the script is listed with other plain text comments to
any other visitor. As it now shares the same origin as Youtube, it can access any
DOM element in the page.

\section {Gazelle}
\subsection{Process Organization}
Every browser instance is in a separate process like Chrome. In addition, within
an instance, each origin has a separate process. The cohesive display in a tab
is achieved by a browser kernel which interacts with disparate processes (up calls).
Plugins in an instance execute as processes separate from browser instances. Thus
failures in plugins does not affect the host page or the browser. These processes
interact with browser kernel through system calls and can do IPC by following the SOP.

\subsection {Landlord Tenant model}
Like many Window managers, plugins use the abstraction of bitmaps to display content
from a web page in a browser instance. Some portions of a page can be delegated by
a landlord to a tenant. When a page displays an advertisement at the side pane, that
portion has been delegated to a tenant page. Landlord can not access DOM tree of
the tenant or the browing history. But it can change the size, position of the
tenant display, replace the URL at the tenant.

\subsection {Display Protection}
Modern browsers allow landlords to overlay a transparent window over a portion
of the browser display. While this has genuine uses, it can be used for click
jacking (tricking the user to click a button on the transparent victim page, when
the user tries to click a visible button the underneath page). Gazelle prevents
click jacking to an extent, by disallowing different origin windows to be
transparently overlaid. This is called the opaque overlay policy. Thus a malicious
webpage can not transparently draw a victim page from a different origin over it.
\end{document}
