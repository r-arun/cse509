
% !TEX TS-program = pdflatex
% !TEX encoding = UTF-8 Unicode

% This is a simple template for a LaTeX document using the "article" class.
% See "book", "report", "letter" for other types of document.

\documentclass[11pt]{article} % use larger type; default would be 10pt

\usepackage[utf8]{inputenc} % set input encoding (not needed with XeLaTeX)

%%% Examples of Article customizations
% These packages are optional, depending whether you want the features they provide.
% See the LaTeX Companion or other references for full information.

%%% PAGE DIMENSIONS
\usepackage{geometry} % to change the page dimensions
\geometry{a4paper} % or letterpaper (US) or a5paper or....
% \geometry{margin=2in} % for example, change the margins to 2 inches all round
% \geometry{landscape} % set up the page for landscape
%   read geometry.pdf for detailed page layout information

\usepackage{graphicx} % support the \includegraphics command and options

% \usepackage[parfill]{parskip} % Activate to begin paragraphs with an empty line rather than an indent

%%% PACKAGES
\usepackage{booktabs} % for much better looking tables
\usepackage{array} % for better arrays (eg matrices) in maths
\usepackage{paralist} % very flexible & customisable lists (eg. enumerate/itemize, etc.)
\usepackage{verbatim} % adds environment for commenting out blocks of text & for better verbatim
\usepackage{subfig} % make it possible to include more than one captioned figure/table in a single float
% These packages are all incorporated in the memoir class to one degree or another...

%%% HEADERS & FOOTERS
\usepackage{fancyhdr} % This should be set AFTER setting up the page geometry
\pagestyle{fancy} % options: empty , plain , fancy
\renewcommand{\headrulewidth}{0pt} % customise the layout...
\lhead{}\chead{}\rhead{}
\lfoot{}\cfoot{\thepage}\rfoot{}

%%% SECTION TITLE APPEARANCE
\usepackage{sectsty}
\allsectionsfont{\sffamily\mdseries\upshape} % (See the fntguide.pdf for font help)
% (This matches ConTeXt defaults)

%%% ToC (table of contents) APPEARANCE
\usepackage[nottoc,notlof,notlot]{tocbibind} % Put the bibliography in the ToC
\usepackage[titles,subfigure]{tocloft} % Alter the style of the Table of Contents
\renewcommand{\cftsecfont}{\rmfamily\mdseries\upshape}
\renewcommand{\cftsecpagefont}{\rmfamily\mdseries\upshape} % No bold!

\makeatletter
   \newcommand\figcaption{\def\@captype{figure}\caption}
   \newcommand\tabcaption{\def\@captype{table}\caption}
\makeatother
%%% END Article customizations

%%% The "real" document content comes below...

\title{CSE 509 Lecture 24}

\author{Prof. Rob Johnson, Scribe:Arun Rathakrishnan}
%\date{} % Activate to display a given date or no date (if empty),
         % otherwise the current date is printed 

\begin{document}
\maketitle
\section {Capability Based Systems}
In addition to Hardware Capabilities (fat pointers) and OS Capabilities (file
descriptors), we have Programming Language Capabilities. Objects should serve
as a capability mechanism. The model is called Object Capability model.\\

The major motivation for capability based systems is to enforce the Principle
of least privileges. A consumer must be granted the least privileges on an object
it needs, in order to perform a particular task. The least privilege depends
on the level of enforcement an underlying machine can provide. For example, when
adding a vector of numbers, a routine needs to only read the values at the pointer
location. But if hardware does not support read only memory locations, we choose
the ability to access only the passed addresses within the bounds, as the least
privilege.\\

Capability based systems must support,
\begin {itemize} \itemsep -2pt
\item Unforgeability.
\item Possession of a handle must mean having access to the resource.
\item Ability to pass capability.
\item Attenaubility - ability to restrict capabilities, as they are passed from
one component to another.
\end {itemize}

Joe-e is a Java like language that makes use of the idea of objects as capabilities.

\subsection {Pitfalls of Ambient Authority}
A system enforcing ambient authority does not identify a source of authority with
a particular permission. It might grant permission for a subject to access an
object, even if the requesting source does not give a permission to do that, but
if another source grants the permission. For example, the billing administrator
gives permission to compiler binary to read billing.txt file, while the user
does not. It is possible for the ambient authority to confuse the 
administrator's permission with the user's and thus wrongly provide access to
the user. We consider how a language supports capability based enforcements in
place of ambient authority. We particularly look at an untrusted module that has 
run in the same address space as trusted code.

\section {Enforcement of Restricted Access in Programming Languages}
Programming languages (Java in particular) enforce restricted access by the below
mechanisms.
\begin {description} \itemsep -2pt
\item [Address Space] Memory safety and Type safety enforced in the JVM prevent
memory errors and isolate separate modules from writing into each other.
\item [Access Specifiers] Public, private, protected access specifiers of classes,
methods and fields, restrict the visibility and hence the usability of components
to a limited scope.
\item [Static Public fields] Can be accessed from any method. They are the sources
of ambient authority. Constructors representing system resources is also one
source of authority, whose exposure/usage must be limited.
\end {description}

\subsection {File class}
For example consider the File class. Let us assume it has been modified to support
least access. We provide only one method that returns a File object, called openAt.
Similar to open\_at system call, it allows only files in the sub-tree of directory
represented by the File object to be accessed.
\begin {verbatim}
    class File{
        public File openAt(String path){
            //check for '..' or absolute path in path
        }
   	}
    class TrustedClass{
       public void init(File f){
            File obj = f.openAt("tmp");
            untrustedObject.doWord(obj);
       }
    }
\end {verbatim}
A method like getParent(), which gives the parent directory will leave room to
overcome the enforcements of this model. Hence such methods should not be available
to untrusted modules.

\subsection {Getting an Object reference in Java}
A method can get Object references only in the below ways.
\begin {description} \itemsep -2pt
\item [Call a Constructor.] We restrict the constructors made available to an
untrusted module, like the File class above.
\item [Returned from a method] other than constructor can return object
references.
\item [Reference is passed as argument.] The trusted code can control this.
\item [Direct access to global static Object.] We can restrict access to be granted
only to immutable final public members. Immutable objects are deeply immutable.
\item [Thrown as an Exception.]
\item [Reflection API] can allow one to access non-public members that could not
be accessed using the standard API. We could modify the Reflection API to prevent
access to certain private/protected members.
\end {description}
By limiting the exposed constructors, and public methods and fields we can ensure
that the untrusted capabilities are limited by the parameters passed to a
untrusted module. As a side effect capability based programming makes code review
of plugin/components easier, to determine which interfaces are exposed by the host
system.

\section {Attenuating Capabilities}
We can wrap objects to create attenuated capabilities. These are especially useful,
if an component needs to pass the capability to another untrusted module. Consider
a ROFile class which allows only the read operation.
\begin {verbatim}
    class ROFile extends File{
        public ROFile(String fname){
            super(fname);
        }
        public read(InputStream i){
            super.read(i);
        }
        public write(OutputStream i) throws InvalidOperationException{
            throw new InvalidOperationException();
        }
    }
\end {verbatim}
This wrapper class will not allow write method on the object. We must ensure
that the method is not final in the parent class. If a method is final, we can
contain a File object in a wrapper class and support all allowed valid method
by calling the same on the contained object. The disallowed operations on File
are simply not performed by a suitable implementation in the wrapper class. APIs
can exclusively use wrapper classes, ensuring that the passed parameters represent
capability.
\end{document}

