
% !TEX TS-program = pdflatex
% !TEX encoding = UTF-8 Unicode

% This is a simple template for a LaTeX document using the "article" class.
% See "book", "report", "letter" for other types of document.

\documentclass[11pt]{article} % use larger type; default would be 10pt

\usepackage[utf8]{inputenc} % set input encoding (not needed with XeLaTeX)

%%% Examples of Article customizations
% These packages are optional, depending whether you want the features they provide.
% See the LaTeX Companion or other references for full information.

%%% PAGE DIMENSIONS
\usepackage{geometry} % to change the page dimensions
\geometry{a4paper} % or letterpaper (US) or a5paper or....
% \geometry{margin=2in} % for example, change the margins to 2 inches all round
% \geometry{landscape} % set up the page for landscape
%   read geometry.pdf for detailed page layout information

\usepackage{graphicx} % support the \includegraphics command and options

% \usepackage[parfill]{parskip} % Activate to begin paragraphs with an empty line rather than an indent

%%% PACKAGES
\usepackage{booktabs} % for much better looking tables
\usepackage{array} % for better arrays (eg matrices) in maths
\usepackage{paralist} % very flexible & customisable lists (eg. enumerate/itemize, etc.)
\usepackage{verbatim} % adds environment for commenting out blocks of text & for better verbatim
\usepackage{subfig} % make it possible to include more than one captioned figure/table in a single float
% These packages are all incorporated in the memoir class to one degree or another...

%%% HEADERS & FOOTERS
\usepackage{fancyhdr} % This should be set AFTER setting up the page geometry
\pagestyle{fancy} % options: empty , plain , fancy
\renewcommand{\headrulewidth}{0pt} % customise the layout...
\lhead{}\chead{}\rhead{}
\lfoot{}\cfoot{\thepage}\rfoot{}

%%% SECTION TITLE APPEARANCE
\usepackage{sectsty}
\allsectionsfont{\sffamily\mdseries\upshape} % (See the fntguide.pdf for font help)
% (This matches ConTeXt defaults)

%%% ToC (table of contents) APPEARANCE
\usepackage[nottoc,notlof,notlot]{tocbibind} % Put the bibliography in the ToC
\usepackage[titles,subfigure]{tocloft} % Alter the style of the Table of Contents
\renewcommand{\cftsecfont}{\rmfamily\mdseries\upshape}
\renewcommand{\cftsecpagefont}{\rmfamily\mdseries\upshape} % No bold!

\makeatletter
   \newcommand\figcaption{\def\@captype{figure}\caption}
   \newcommand\tabcaption{\def\@captype{table}\caption}
\makeatother
%%% END Article customizations

%%% The "real" document content comes below...

\title{CSE 509 Lecture 3}

\author{Prof. Rob Johnson, Scribe:Arun Rathakrishnan}
%\date{} % Activate to display a given date or no date (if empty),
         % otherwise the current date is printed 

\begin{document}
\maketitle

\section{Review}
\subsection{Standard as a trusted entity}
A trusted entity can also be an implementation of a standard or even a standard by itself.
It is possible that both the entities can violate trust. For example, the inclusion of
Null encryption in protocols, may be viewed as an attempt to violate trust.

\subsection{Protocol Downgrade Attack}
In this attack, one side of communication is forced to lower the level of encryption,
making it easier to violate trust and attack the system. \\
\\
For example, consider a banking site that supports https and http protocols. If your 
browser allows you to connect using http through an unsecure network, it may be subject to
a man-in-the-middle attack. The attacker can talk to the victim using http, while using in
formation from that communication to talk in https to the bank. The bank sends https
response to the attacker, which can be converted on the fly to http and presented to the
victim.\\
\\
Here the attacker successfully lowers the level of encryption in his communication with
the victim. Modern browsers can be forced to use SSL to establish https connections with
a website during the first time, the browser visits the site. This can help to prevent
this attack.

\section{Access Control}
Access control is a mechanism by which a service provide is able to provided a trusted 
service to a client it does not trust. The service provider,

\begin{itemize} \itemsep -2pt  % reduce space between items
\item receives request from a client.
\item Needs to identify which clients can perform which actions.
\item Needs to implement the decisions.
\end{itemize}

The below are three access control mechanisms.
\begin{itemize} \itemsep -2pt  % reduce space between items
\item Access Control Matrix (Role Based Access Control)
\item Access Control Lists
\item Bell,  Lapadula/ Biba Model
\end{itemize}

We will consider a kernel as a service provider which needs to service requests to its
file system. The common operations allowed are read and write for files; lookup, create,
link, unlink (remove), etc for directories. The kernel may also support super-block
operations.

\section{Access Control Matrix}
The access control matrices control - who (subject) can do what (operation/permission) to
what (object). The subject is typically a user (in a multi-user OS), or an application (in
an Android kernel). We consider the Harrison Rizzo Ullman model, where each row in the
matrix corresponds to a subject - a single user. User groups are not considered. Each 
column corresponds to an object and each cell to the operation allowed for a subject on a
particular object.You can add subjects and columns to the matrix.\\
\\
The access control matrix, irrespective of its representation, is an object whose access
needs to be controlled too. So it has an entry as to who can change the operations for
a given object, identifying the subject as an owner. In unix file system, a file's owner can
change the operations for otherusers (we do not consider groups here). \\
\\
In addition to the matrix, the model contains rules which can be used to modify the
matrix. The rules are typically listed in \texttt{Pre-Condition: Operation} format. A
rule is executed if the precondition (based on the access control matrix data) is satisfied,
which results in permission bits of a user being changed for a file.
\subsection{Example rule for chmod}
\begin{verbatim}
     chmod(executor, object, user, permission):
         if(owner is in A[executor][object]):
             A[user][object] <- permission
\end{verbatim}

\subsection{Example rule for chown}
Only a root user can change the ownership of a file in unix filesystems.

\begin{verbatim}
    chown(executor, object, newOwner):
            if(executor == owner):
                 Remove the owner permission from previous owner of object
                 A[newOwner][object] <- owner
\end{verbatim}
\subsection{HRU Theorem}
Given an initial state of the access control matrix and a list of rules for updating the
 matrix and a final state of the access control matrix, to determine whether the application
of rules on the list takes the matrix from  initial state to final state is undecidable.\\
\\
As a result, we can never figure out if a known, but undesirable security violation can be
averted by using an access control matrix through a sequence of updates. Typically,
most subjects do not have access to most objects and hence, the matrix is too sparse to
be used in practice. Access Control Lists are more practical. They store for each object,
only subjects with valid permissions and the access rights. Roughly, a column of access
control matrix corresponds to Access Control List for that object.

\section{Monotonicity/Order Dependence}
We assume that groups can contain a subset of users, and can appear as subjects in
access control lists. The permission for an object for any subject now may consider
user permissions and a combination of several groups' permission to which a user can 
belong.

Some of the policies can be order dependent like,

\begin{itemize} \itemsep -2pt  % reduce space between items
\item first match policy
\item last match policy
\end{itemize}

Some of them can be order independent like,
\begin{itemize} \itemsep -2pt  % reduce space between items
\item most specific/ most general
\item most permissive/ most restricted
\end{itemize}

\subsection{Most specific policy in Unix}
Unix file systems check for a file if,the user is an owner or has user permissions in ACL
or belongs to a group which is listed as a subject in ACL or if not, finally considers the
other permission bits.

\subsection{Most permissive policy in Unix}
Assume Unix has two lists as a substitute for ACL. The first list is a list of users for
files. The second is the list of groups for a file. It can search through all the valid
subjects, (user's own entry or any of user's group in ACL for that file) and select the
most permissive policy for that file.

\subsection{Negative Entries}
A permission bit may be a negative entry, which expressly prevents an operation on an
object. IP Tables, Windows Files System, Firewalls use such entries in their ACL. Most
restrictive or most specific policy rules may be enforced strictly. It is also possible
to expressly prevent some operations on objects by a subject without using negative entries.

\section{Bell/ Lapadula}
Objects and subjects are classified based on a level and a set called compartment. Levels
are usually tagged as,

\begin{itemize} \itemsep -2pt  % reduce space between items
\item Top Secret
\item Secret
\item For Official Use Only (FOUO)
\item Public
\end{itemize}

A compartments form subsets of information classification that further restrict access
within a level. Compartments apply across all levels.\\
\\
An object labelled with level and compartment $(L, CS)$ can be accessed by a subject
with level and compartment $(L^{\prime}, CS^{\prime})$ only if,  $L \leq L^{\prime}$ and
$CS \subset CS^{\prime}$.

\section{Mandatory Access Control Systems}
An access control system is discretionary if the subjects are able to determine or change
the access level of objects. In a mandatory access control system, no user can determine
or change the access level of any object. For example, you may combine a Top Secret document
and a Secret document to form a new document with level Secret in a discretionary access
control. But in mandatory access control, the system will determine the level to be
Top Secret, which can not be changed by subjects. Such a policy is called  a "No write
down" policy.

\end{document}
