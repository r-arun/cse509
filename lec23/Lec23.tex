
% !TEX TS-program = pdflatex
% !TEX encoding = UTF-8 Unicode

% This is a simple template for a LaTeX document using the "article" class.
% See "book", "report", "letter" for other types of document.

\documentclass[11pt]{article} % use larger type; default would be 10pt

\usepackage[utf8]{inputenc} % set input encoding (not needed with XeLaTeX)

%%% Examples of Article customizations
% These packages are optional, depending whether you want the features they provide.
% See the LaTeX Companion or other references for full information.

%%% PAGE DIMENSIONS
\usepackage{geometry} % to change the page dimensions
\geometry{a4paper} % or letterpaper (US) or a5paper or....
% \geometry{margin=2in} % for example, change the margins to 2 inches all round
% \geometry{landscape} % set up the page for landscape
%   read geometry.pdf for detailed page layout information

\usepackage{graphicx} % support the \includegraphics command and options

% \usepackage[parfill]{parskip} % Activate to begin paragraphs with an empty line rather than an indent

%%% PACKAGES
\usepackage{booktabs} % for much better looking tables
\usepackage{array} % for better arrays (eg matrices) in maths
\usepackage{paralist} % very flexible & customisable lists (eg. enumerate/itemize, etc.)
\usepackage{verbatim} % adds environment for commenting out blocks of text & for better verbatim
\usepackage{subfig} % make it possible to include more than one captioned figure/table in a single float
% These packages are all incorporated in the memoir class to one degree or another...

%%% HEADERS & FOOTERS
\usepackage{fancyhdr} % This should be set AFTER setting up the page geometry
\pagestyle{fancy} % options: empty , plain , fancy
\renewcommand{\headrulewidth}{0pt} % customise the layout...
\lhead{}\chead{}\rhead{}
\lfoot{}\cfoot{\thepage}\rfoot{}

%%% SECTION TITLE APPEARANCE
\usepackage{sectsty}
\allsectionsfont{\sffamily\mdseries\upshape} % (See the fntguide.pdf for font help)
% (This matches ConTeXt defaults)

%%% ToC (table of contents) APPEARANCE
\usepackage[nottoc,notlof,notlot]{tocbibind} % Put the bibliography in the ToC
\usepackage[titles,subfigure]{tocloft} % Alter the style of the Table of Contents
\renewcommand{\cftsecfont}{\rmfamily\mdseries\upshape}
\renewcommand{\cftsecpagefont}{\rmfamily\mdseries\upshape} % No bold!

\makeatletter
   \newcommand\figcaption{\def\@captype{figure}\caption}
   \newcommand\tabcaption{\def\@captype{table}\caption}
\makeatother
%%% END Article customizations

%%% The "real" document content comes below...

\title{CSE 509 Lecture 22}

\author{Prof. Rob Johnson, Scribe:Arun Rathakrishnan}
%\date{} % Activate to display a given date or no date (if empty),
         % otherwise the current date is printed 

\begin{document}
\maketitle
\section {Capability Based Systems}
Capability based systems provide a mechanism for granting acccess to resources
through,
\begin {itemize} \itemsep -2pt
\item providing unforgeable references/handles.
\item possessing handle equates to having access to resource.
\item handles may also specify operations that a holder can perform on the resource.
\end {itemize}

Usually the operating system can be counted upon to provide references that are
representative of some actions on a resource and ensure that the reference can not
be forged by a user process.

\section {Hardware Capability Based Systems}
In native client systems, the untrusted code can be given more fine grained access
to resources in the trusted region if the hardware supports fat pointers and typed/ tagged
memory (which differentiates a pointer data from non-pointer data in the memory).
The OS has the capability to provide boundaries of stack, heap and data section
in pre determined register locations and provides a restrict instruction for
selecting a smaller pointer address range within the valid range.
\begin {verbatim}
     store 0, A
     store 1<<64, A+1
     store p, A+2
     load A, %p0
\end {verbatim}
\\
This code causes assigning the valid address range to entire pointer range. But 
the addresses $A$ to $A+2$ are tagged as non-pointers which will cause the load
instruction into a pointer type register to fail.\\

With such a system in place a trusted entity can pass pointers to addresses in
trusted region, to an untrusted code. The untrusted code can access addresses
which are provided as arguments, respecting the bounds placed by the fat pointers
but not any other region in the trusted entity. The caller can specifiy if the
pointers when deferenced can be read from or written to by extra permissions,
passed alongside the arguments in the call.

\section {Capability Based Systems used by Operating System}
\subsection {Passing file descriptors across processes}
A process can pass a file descriptor in its file descriptor table to another
process, via the kernel through pipes, which causes the second process to
incorporate the file descriptor entry in its own file descriptor table. Expect
for the file descriptor index, both processes have the copied file descriptor
representing identical file state. It should be noted that the permissions obtained
in the copied file descriptor, may not be achieved through a regular open by the
second process.

\subsection {File Descriptors in Unix}
With respect to capability, we can evaluate the capability enforced by file
descriptors based on,
\begin {description} \itemsep -2pt
\item {Forgeability} \begin {itemize} \itemsep -2pt \item Can only be created by
the operating system.
\item Can not be accessed across address spaces or modified by a process bypassing
the kernel.
\end {itemize}
\item {Holding Reference}
\begin {itemize} \itemsep -2pt 
\item Specify access rights of a file completely. No other information is required
further for accessing a file, once a file descriptor is held by a process.
\end {itemize}
\item {Delegating Capability}
\begin {itemize} \itemsep -2pt 
\item Can be delegated through pipes as noted above.
\end {itemize}
\end {description}

\subsection {Confused Deputy Problem}
Consider an operating system which enforces unix like file access permissions
for users and groups. In addition, an administrator can set a list of files
that an executable file can open. In this setting, consider a compiler program
which takes a source file as input, and writes the compiled binary to an
output file. The user running the compiler program must have permissions to
access, the input and output files. The compiler also writes usage information
to a billing file, for which the permission is granted by an administrator. No
user of the compiler is granted access to write the billing file.

\begin {verbatim}
     compile(){
         fd1 = open(input_arg, "r");
         fd2 = open(output_arg, "w");
         read(fd1, ...);
         process(fd1);
         truncate(fd2);
         write(fd2, ...);
         fd = open("billing_file", "w");
         write(fd, ...);
    }
\end {verbatim}

For successful compilation the user must have correct permissions to input\_arg
and output\_arg. Even though the user does not have permission to write to
"billing\_file", during open OS accepts the extra permissions granted to compiler
binary, to open and write to that file.\\

Assume that a user passes, "billing\_file" as output\_arg. Now the OS confuses
the extra permissions granted to compiler binary, with the permissions of the user
and lets the file be truncated, leading to the actual billing\_file being overwritten.\\

The compiler program serves as a deputy to two masters, users - who use the
compiler, and administrators - who perform the billing. Here the deputy used the
permissions granted by one master (access to billing file), to perform an
unauthorised operation (user does not have access to billing file) for another
master.
\subsection {Establishing capabality boundaries through exec}
exec system call flushes the address space of a process and overwrites it with
code in the argument provided. However, it does not change the file table
descriptor. Consider the code snippet below.

\begin {verbatim}
    fd1 = open(...);
    fd2 = open(...);
    exec(p2, "--input1 = fd1", "--input2 = fd2");
\end {verbatim}

The process p2 may not have access to fd1 and fd2, but on exec it can use the
descriptors with the same permissions the original process possessed. In terms
of capability, having access to file descriptors means that they can be used
by p2. We can use this approach to solve the confused deputy problem.\\

Here's the code for "compiler.c" which opens the input and output files, only
if the user possesses the permission to do so. This binary does not have special
permission to open "billing\_file".

\begin {verbatim}
    fd1 = open(input_arg, "r");
    fd2 = open(output_arg, "w");
    if(fd1 < 0 || fd2 < 0) exit(1);
    exec("_compiler", "--input = fd1", "--output = fd2");
\end {verbatim}

Here's the code for "\_compiler.c" whose binary is granted write access to file
descriptor of "billing\_file" through a special mechanism, different from open
system call. The code is execed and can access file descriptors fd1 and fd2. It
performs the actual compilation and writes to the billing file.

\begin {verbatim}
    fd1 = parse_args(...);
    fd2 = parse_args(...);
    billing_fd = get_fd(...);
    ...
    read(fd1);
    ...
    write(fd1);
    write(billing_fd, ...);
\end {verbatim}

By separating the permission granted by different masters across different permission
domains, we can solve this instance of confused deputy problem. The compiler may
need to read several include files, and library code independent of user input.
A capability based system can provide a similar mechanism, different from open
for the compiler to read these files, thus making the compiler code, independent
of the ambient security model enforced by the file system.

\subsection {Secure file dialog}
Based on the ability to pass file descriptors, we can allow untrusted applications
that can not call open system call directly, to access files through a Secure
File Dialog server. The secure file dialog server receives requests from the
untrusted application and then shows a prompt to user asking if access to the
file can be provided. If the user agrees, the server then opens the file (it has
the necessary permissions) and passes the file descriptor through pipe to the
untrusted application. Here too, by just holding file descriptors, the untrusted
applications gains capability to access the desired file.
\section {Eliminating Ambient Authority}
Consider a scenario in which a process is provided capability to access only parts
of the file system indicated by file descriptors. We can assume that only files
in the subtree of the file descriptors (obtained through a open like mechanism) at
the start of the process can be accessed. We can borrow ideas from open\_at system
call.
\begin {verbatim}
	open_at (dir_fd, path, ...);
\end {verbatim}

path is a relative path, present in the path represented by dir\_fd, which provides
the file descriptor of a parent directory. If open\_at can enforce that file descriptors
returned by it are strictly in the subtree of dir\_fd path, then we can use it for
supporting the intended capability based system. There are two general principles
that must be followed in capability based systems.

\begin {itemize} \itemsep -2pt
\item Global access to resources must be prevented from being held.
\item Capability leaks must be avoided.
\end {itemize}

Thus we must find a way to check paths which are absolute (global access) and use
".." to navigate to parent directory (capability leak).
\end{document}

