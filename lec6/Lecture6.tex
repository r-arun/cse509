
% !TEX TS-program = pdflatex
% !TEX encoding = UTF-8 Unicode

% This is a simple template for a LaTeX document using the "article" class.
% See "book", "report", "letter" for other types of document.

\documentclass[11pt]{article} % use larger type; default would be 10pt

\usepackage[utf8]{inputenc} % set input encoding (not needed with XeLaTeX)

%%% Examples of Article customizations
% These packages are optional, depending whether you want the features they provide.
% See the LaTeX Companion or other references for full information.

%%% PAGE DIMENSIONS
\usepackage{geometry} % to change the page dimensions
\geometry{a4paper} % or letterpaper (US) or a5paper or....
% \geometry{margin=2in} % for example, change the margins to 2 inches all round
% \geometry{landscape} % set up the page for landscape
%   read geometry.pdf for detailed page layout information

\usepackage{graphicx} % support the \includegraphics command and options

% \usepackage[parfill]{parskip} % Activate to begin paragraphs with an empty line rather than an indent

%%% PACKAGES
\usepackage{booktabs} % for much better looking tables
\usepackage{array} % for better arrays (eg matrices) in maths
\usepackage{paralist} % very flexible & customisable lists (eg. enumerate/itemize, etc.)
\usepackage{verbatim} % adds environment for commenting out blocks of text & for better verbatim
\usepackage{subfig} % make it possible to include more than one captioned figure/table in a single float
% These packages are all incorporated in the memoir class to one degree or another...

%%% HEADERS & FOOTERS
\usepackage{fancyhdr} % This should be set AFTER setting up the page geometry
\pagestyle{fancy} % options: empty , plain , fancy
\renewcommand{\headrulewidth}{0pt} % customise the layout...
\lhead{}\chead{}\rhead{}
\lfoot{}\cfoot{\thepage}\rfoot{}

%%% SECTION TITLE APPEARANCE
\usepackage{sectsty}
\allsectionsfont{\sffamily\mdseries\upshape} % (See the fntguide.pdf for font help)
% (This matches ConTeXt defaults)

%%% ToC (table of contents) APPEARANCE
\usepackage[nottoc,notlof,notlot]{tocbibind} % Put the bibliography in the ToC
\usepackage[titles,subfigure]{tocloft} % Alter the style of the Table of Contents
\renewcommand{\cftsecfont}{\rmfamily\mdseries\upshape}
\renewcommand{\cftsecpagefont}{\rmfamily\mdseries\upshape} % No bold!

\makeatletter
   \newcommand\figcaption{\def\@captype{figure}\caption}
   \newcommand\tabcaption{\def\@captype{table}\caption}
\makeatother
%%% END Article customizations

%%% The "real" document content comes below...

\title{CSE 509 Lecture 6}

\author{Prof. Rob Johnson, Scribe:Arun Rathakrishnan}
%\date{} % Activate to display a given date or no date (if empty),
         % otherwise the current date is printed 

\begin{document}
\maketitle

\section{Return Oriented Programming}
\begin {itemize} \itemsep -2pt
\item Find small gadgets in binary code. 
\item Compose gadgets together to accomplish work. 
\end {itemize}

We can do this by placing the address of the next gadget to be called after the
arguments to the current gadget. At the end of execution of the gadget, SP will
point to the next gadgets address and machine will consider it as the caller
and transfer control there by pushing the SP contents to IP. Gadgets have a small
number of instructions one after another and end with a return, which can help
identify them in a library's binary.

\subsection{General Techniques for Arbitrary Computation}
There are enough gadgets in libc to make a machine perform any arbitrary
computation.\\
\\Consider the two gadgets that can load from an arbitrary address to a register
and store to an address from a register.
Load into $\%eax$, addr.
\begin{verbatim}
     pop %eax
     ret
\end{verbatim}
Here the SP points to bytes containing the address of the argument. This address
is popped  to $\%eax$ register.\\
\\
Store addr, $\%eax$
\begin{verbatim}
     pop %edx
     mov (%edx), %eax
     ret
\end{verbatim}
Here SP contains the address where contents of $\%eax$ must be stored. This
gadget pops SP's contents to $\%edx$. To the address contained in $\%edx$ the
contents of $\%eax$ are moved.\\
\\
It is possible to string together gadgets in many libraries (to perform increment, decrement, loops,
test equivalence, etc) to create machine code powerful enough to perform any computation.

\subsection{Example Attack}
\begin {itemize} \itemsep -2pt
\item Use gadgets to call  mmap.
\item Call mmap to create pages in address space that are writeable and executable.
\item Jump to the address created using mmap.
\end {itemize}

This helps an attacker overcome the NX bit that prevents executing code on the
stack. In Just in Time compilers, the byte code is converted and written to
pages in memory as machine code. Then the control is transferred to the
translated pages and execution is performed. This is an example of a scenario
where both Write and Execute permissions are enabled for a file.

\section {Address Space Layout Randomization}
It is a defense  against all kinds of buffer flow attacks, such as
\begin {itemize} \itemsep -2pt
\item Code Injection - By Stack Section randomization
\item Return to libc - Code section randomization
\item Return oriented programming - Can not locate gadgets within libraries at random locations.
\end {itemize}
\subsection {Stack Randomization Technique}
The position of the stack for a process is maintained in the Stack Pointer. SP
is only incremented or decremented, but there is no instruction that causes SP
to jump to a different value. Since a program just needs SP to keep track of
the activation records, this technique moves the entire stack to different
location and updates the stack pointer. This is done when the binary code of
the process is loaded to memory, when the program is started and stack is setup.
The stack's position does not change when a process is executing.\\

During code injection an attacker overwrites the return address of the vulnerable
function to specify the address of the malicious program on the stack, to which
the control should jump. Due to randomization, the address where process's stack
begins and hence the address of malicious program changes during every execution,
thus preventing execution of the intended malicious code with high likelihood.\\

When only stack randomization is supported, return to libc attack becomes more
difficult but not unlikely. The argument to the function can not be passed as
the address of the buffer in stack has been shifted due to randomization. One
way to subvert this is to provide as input the intended attack string to a
program that is known to use heap, several times. This fills the heap section
(assumed to be shared among processes) and a random address is likely to point
 to the heap. Filling the string with spaces at beginning can help the attacker
quickly succeed and carry out the attack.

\subsection {Effective Address Space Layout Randomization}
In addition to stack section randomization, the following are done to effectively
implement address space randomization.

\begin {itemize} \itemsep -2pt
\item Randomize location of all libraries (they must be page aligned).
\item Randomize location of heap.
\item Randomize mmaped location.
\end {itemize}

Randomizing the text section and location of libraries is effective in preventing
return to libc attack and return oriented programming. In the later case, the
attacker can not scavenge and look for gadgets to string together to execute
arbitrary code as its position in the address space is not fixed.

\subsection {Limitations of ASLR in 32 bit machines}
\begin {itemize} \itemsep -2pt
\item Moving around libraries that are forced to be page aligned may lead to
fragmentation of virtual memory address space.
\item In a 32 bit machine there is limited room for randomization (about $2^{16}$)
different starting addresses are only possible for a given section.
\item Locating this may lead to several crashes that can alert a system
administrator. It has been shown that the vulnerability can be exploited in
about two hours in a 32 bit machine.
\item 64 bit machines have large address spaces that can support ASLR and reduce
significantly the success of an attack compared to 32 bit machines.
\end {itemize}

\subsection {Limitations of ASLR in Windows OS}
Linux supports position independent code ($pic$), that lets a process run a library
code irrespective of its position in the address space. This was done to prevent
the clash of libraries that hard-coded themselves to specific locations. This
has been extended to support ASLR.\\
\\
In case of Windows, there are different options for a user to choose from based
on config options.
\begin {itemize} \itemsep -2pt
\item ASLR on.
\item ASLR Opt-in : Libraries declare in header to say if they can be loaded at
different locations.
\item ASLR Opt-out: Libraries declare in header to say if they opt out from ASLR.
\end {itemize}

Libraries that prevent ASLR and force the OS to load them at specific locations
are good sources of gadgets that can be used to perform return oriented programming
attacks. The lack of randomization enforcement has led to attacks on Adobe
Acrobat Reader by exploitation of gadgets within libraries that must be loaded at
a specific location in the address space.
\end{document}
