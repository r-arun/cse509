
% !TEX TS-program = pdflatex
% !TEX encoding = UTF-8 Unicode

% This is a simple template for a LaTeX document using the "article" class.
% See "book", "report", "letter" for other types of document.

\documentclass[11pt]{article} % use larger type; default would be 10pt

\usepackage[utf8]{inputenc} % set input encoding (not needed with XeLaTeX)

%%% Examples of Article customizations
% These packages are optional, depending whether you want the features they provide.
% See the LaTeX Companion or other references for full information.

%%% PAGE DIMENSIONS
\usepackage{geometry} % to change the page dimensions
\geometry{a4paper} % or letterpaper (US) or a5paper or....
% \geometry{margin=2in} % for example, change the margins to 2 inches all round
% \geometry{landscape} % set up the page for landscape
%   read geometry.pdf for detailed page layout information

\usepackage{graphicx} % support the \includegraphics command and options

% \usepackage[parfill]{parskip} % Activate to begin paragraphs with an empty line rather than an indent

%%% PACKAGES
\usepackage{booktabs} % for much better looking tables
\usepackage{array} % for better arrays (eg matrices) in maths
\usepackage{paralist} % very flexible & customisable lists (eg. enumerate/itemize, etc.)
\usepackage{verbatim} % adds environment for commenting out blocks of text & for better verbatim
\usepackage{subfig} % make it possible to include more than one captioned figure/table in a single float
% These packages are all incorporated in the memoir class to one degree or another...

%%% HEADERS & FOOTERS
\usepackage{fancyhdr} % This should be set AFTER setting up the page geometry
\pagestyle{fancy} % options: empty , plain , fancy
\renewcommand{\headrulewidth}{0pt} % customise the layout...
\lhead{}\chead{}\rhead{}
\lfoot{}\cfoot{\thepage}\rfoot{}

%%% SECTION TITLE APPEARANCE
\usepackage{sectsty}
\allsectionsfont{\sffamily\mdseries\upshape} % (See the fntguide.pdf for font help)
% (This matches ConTeXt defaults)

%%% ToC (table of contents) APPEARANCE
\usepackage[nottoc,notlof,notlot]{tocbibind} % Put the bibliography in the ToC
\usepackage[titles,subfigure]{tocloft} % Alter the style of the Table of Contents
\renewcommand{\cftsecfont}{\rmfamily\mdseries\upshape}
\renewcommand{\cftsecpagefont}{\rmfamily\mdseries\upshape} % No bold!

\makeatletter
   \newcommand\figcaption{\def\@captype{figure}\caption}
   \newcommand\tabcaption{\def\@captype{table}\caption}
\makeatother
%%% END Article customizations

%%% The "real" document content comes below...

\title{CSE 509 Lecture 25}

\author{Prof. Rob Johnson, Scribe:Arun Rathakrishnan}
%\date{} % Activate to display a given date or no date (if empty),
         % otherwise the current date is printed 

\begin{document}
\maketitle
\section {Case Study: Algorithmic Complexity Attacks for UNIX File System Race Conditions}
The above paper describes how different systems can interact to create vulnerabilities
and how the combination of race condition, unix system call implementation, scheduling and cache 
systems can be easily exploited by an attacker to perform an unauthorised file
access.

\section {Race Condition}
Consider a victim program and an attacker script that run side by side on a
processor. The operating system scheduler can select a process to run for a
specific time slice.

Here's the victim program, victim.c.
\begin{verbatim}
    fd1 = open(filename,...);
    ...
    fd2 = open(filename,...);
\end{verbatim}

Here's the attacker's script.
\begin{verbatim}
    rm filename
   	ln -s target filename
\end{verbatim}
After the first open system call in victim.c completes, the process may be
scheduled out, and it is possible for the attacker's script to get scheduled.
If that happens, and the attacker script completes before the second open, it
would turn out that $fd1$ and $fd2$ are in fact descriptor's to different files,
the latter corresponding to the file pointed by the newly created soft link.\\

In practice $access$ system call may be used in conjunction with $open$. $access$
checks if the real user-id of the current process has permissions to open a file
for certain operations. It is especially useful in programs which use $setuid(0)$
to gain root access, but still want to access files with the same permission as
the process's real user possesses.
\begin{verbatim}
    if(access(filename, ...)){
        open(filename,...);
    }
\end{verbatim}
$lpr$ command which allows a file to be sent to a printer, used a similar approach.
It is possible for an attacker to exploit $lpr$ as follows. The attacker can
create a file for which the $access$ call would succeed and start lpr. After
the $access$ call succeeds, but before $open$ is called, if the attacker's
script gets scheduled and deletes the file and creates a link, as in the
previous example, attacker will be able to access /etc/shadow file, which is
protected with root permissions.

\begin{verbatim}
    ln -s innocent_file filename
    lpr filename arg
   	rm filename
    ln -s /etc/shadow filename
\end{verbatim}

\section {Counter measures}
\subsection {Eliminate access system call}
We could allow the process to reset the effective user id before opening a file.
But if the effective user id is set from $0$ to another value, operating system
can not reset it back to $0$, again using $setuid$. But the OS provides a fix
in the form of a $setres$ system call, which remembers all previous values of
effective user id, and allows the process to set a effective user id that a
process possessed before.

\subsection {Prevent scheduling till open returns}
The only way for an attack to succeed is to make changes to the file system,
between the execution of $access$ and $open$. If we could prevent scheduling, after
$access$ is called but before $open$ returns, we can effectively prevent this
 attack. But it turns out that each look up operation, which maps file name
 to inode on disk incurs I/O. It has been shown that for long paths, it could take several minutes to resolve
the path and get an inode. Preventing scheduling can effectively freeze the
system in such cases.

\subsection {Transaction Based System}
Such a system would begin a transaction before the $access$ call and check that
the inode object being referred to is the same during both the calls. If they are the
same, the transaction is committed, otherwise it does not proceed.

\subsection {Repeating checks}
Though it may seem unlikely that an attacker can exploit race conditions to
mount a successful attack, by choosing the input parameters (long file names),
time of attack (not many other processes are running alongside), one can better
the odds. By repeating access check and open, ensuring that the unique identifiers
of the file remain the same during each attempt, we can make it difficult for
an attack to succeed. Instead of having to succeed once, the attacker must
make changes to the file systems, $d$ times between system calls to successfully
exploit the vulnerability.
\begin{verbatim}
    for(i = 0 to d-1){
         if(access(filename, ...)){
             fd[i] = open(filename,...);
             st[i] = fstat(fd[i]);
             assert( i == 0 || st[i-1] == st[i]);
         }
    }
\end{verbatim}

This fix is effective because, the open should be to the same file during all
attempts, otherwise the assertion will fail. This also means that cache effects
ensure that subsequent $open$'s succeed fast, possibly eliminating I/O and hence
making it highly likely that after $open$ returns, $access$ gets called before
the current process is scheduled out. \\

The problem with this approach is that, the attacker controls file name and
by setting the softlink to different long paths that point to the same inode,
he can make sure that open system call incurs I/O, despite having to open the
same file $d$ times.

\subsection {Atomic Checks}
The major flaw with the previous approach is that, there exists different paths to
the same file name, and open tries to fetch an entire long path in one go. This
buys time for the attacker's code to be scheduled, during each time the check is
performed.
\begin{verbatim}
foreach atom in filename{
    for(i = 0 to d-1){
         if(access(atom, ...)){
             fd[i] = open(atom,... |O_NOFOLLOW);
             st[i] = fstat(fd[i]);
             assert( i == 0 || st[i-1] == st[i]);
     	 }
    }
}
\end{verbatim}
By preventing the open system call from following a link and incrementally opening
a file path on a component by component basis, we can ensure that open executes
fast. \\

It is only possible to change the file name between $access$ and $open$, if the
time slice for the process expires at the end of a system call, and the attacker's
process can be scheduled between these times, especially if it has a higher priority
than the victim process. In unix, if a system call has started before the end of
a time slice, it can continue running beyond the time slice, provided it makes
no I/O. But having to precisely schedule a process (by using $sleep$ system call)
 and have it succeed multiple times, makes such an attack unlikely to succeed.

\subsection {Algorithmic Attacks}
Linux file system makes use of dcache as a in-memory hash table, which given a 
directory id, can return a inode for a file. Whenever $lookup$ has to be performed
during the $open$ call, the dcache is searched to get the inode or dentry of recently
accessed files. When multiple keys collide on the same slot in the hash table,
chaining is used to resolve collision. This leaves the possibility of having a
long chain of cached entries for the $lookup$ call to search through, which can
take longer than a scheduling time slice, thus eliminating any gains made through
making lookups for short atomic paths during $open$ system call. The attacker can thus
increase the interval between $open$ and $access$, and hope to schedule his
process successfully between the two.

\subsection {Birthday Attacks}
Several operating systems like Linux, Open BSD, Solaris use similar dcache
mechanisms. Old versions of perl had collision bugs that could be exploited to
stage DoS attacks on webservers. The general hashing scheme used for dcache is
\[
hash(dirid, name) = h_2(dirid, h(name))
\]
If the $name$'s hash to the same slot, by placing them in the same directory,
an attacker can ensure that they will have the same $hash$ value. The $h$ function
is defined as below.
\[
h(s_{n-1}... s_0) = \sum \limits_{i=0}^{n-1} s_i 17^i \bmod {2^{32}} = \sum \limits_{i=0}^{\frac {n-1}{2}} s_i 17^i + \sum \limits_{i=\frac{n-1}{2}-1}^{n-1} s_i 17^i
\]
Thus,
\[
h(s_{n-1}... s_0) = h(s_{\frac {n}{2}} ... s_0) + 17^{\frac {n-1}{2}+1} h(s_{n-1}...s_{\frac {n-1}{2}+1}).
\]
\\
If we can find two strings $S_1$ and $S_2$ such that
$h(S_1) = -17^{\frac {n-1}{2}+1} h(S_2) \bmod {2^{32}}$, we can ensure that both
strings have the same slot. This attack is called "Birthday Attack". Assuming 
that slot size is a $32$ bit number, the total number of values a hash function
can match is $2^{32}$. If we can generate $k$ random strings, compute their
hash and store it in a table and sort the table by the hash key, by birthday
paradox,
\[
P[match] = \frac {2^k} {2^{32}}
\]
\[
E[number of matches] ={2^k} \times \frac {2^k} {2^{32}}
\]

By choosing an appropriate value for $k$, we can expect to have several names
from among the randomly generated strings with matching $h$ values, which can
be used to cause collisions in dcache.
\begin {itemize} \itemsep -2pt
\end{document}

