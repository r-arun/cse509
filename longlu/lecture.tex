% !TEX TS-program = pdflatex
% !TEX encoding = UTF-8 Unicode

% This is a simple template for a LaTeX document using the "article" class.
% See "book", "report", "letter" for other types of document.

\documentclass[11pt]{article} % use larger type; default would be 10pt

\usepackage[utf8]{inputenc} % set input encoding (not needed with XeLaTeX)

%%% Examples of Article customizations
% These packages are optional, depending whether you want the features they provide.
% See the LaTeX Companion or other references for full information.

%%% PAGE DIMENSIONS
\usepackage{geometry} % to change the page dimensions
\geometry{a4paper} % or letterpaper (US) or a5paper or....
% \geometry{margin=2in} % for example, change the margins to 2 inches all round
% \geometry{landscape} % set up the page for landscape
%   read geometry.pdf for detailed page layout information

\usepackage{graphicx} % support the \includegraphics command and options

% \usepackage[parfill]{parskip} % Activate to begin paragraphs with an empty line rather than an indent

%%% PACKAGES
\usepackage{booktabs} % for much better looking tables
\usepackage{array} % for better arrays (eg matrices) in maths
\usepackage{paralist} % very flexible & customisable lists (eg. enumerate/itemize, etc.)
\usepackage{verbatim} % adds environment for commenting out blocks of text & for better verbatim
\usepackage{subfig} % make it possible to include more than one captioned figure/table in a single float
% These packages are all incorporated in the memoir class to one degree or another...

%%% HEADERS & FOOTERS
\usepackage{fancyhdr} % This should be set AFTER setting up the page geometry
\pagestyle{fancy} % options: empty , plain , fancy
\renewcommand{\headrulewidth}{0pt} % customise the layout...
\lhead{}\chead{}\rhead{}
\lfoot{}\cfoot{\thepage}\rfoot{}

%%% SECTION TITLE APPEARANCE
\usepackage{sectsty}
\allsectionsfont{\sffamily\mdseries\upshape} % (See the fntguide.pdf for font help)
% (This matches ConTeXt defaults)

%%% ToC (table of contents) APPEARANCE
\usepackage[nottoc,notlof,notlot]{tocbibind} % Put the bibliography in the ToC
\usepackage[titles,subfigure]{tocloft} % Alter the style of the Table of Contents
\renewcommand{\cftsecfont}{\rmfamily\mdseries\upshape}
\renewcommand{\cftsecpagefont}{\rmfamily\mdseries\upshape} % No bold!

\makeatletter
   \newcommand\figcaption{\def\@captype{figure}\caption}
   \newcommand\tabcaption{\def\@captype{table}\caption}
\makeatother
%%% END Article customizations

%%% The "real" document content comes below...

\title{Android Application Vetting}

%\date{} % Activate to display a given date or no date (if empty),
         % otherwise the current date is printed 

\begin{document}
\maketitle
\section {Comdroid}
\begin {itemize} \itemsep -2pt
\item Talks about Android components.
\item Interactions through intents. Application processes run in separate
protection domains. Intents are like IPC messages. Android framework takes
care of unmarshalling arguments and invoking the call in target app.
\item Threat Model: Applications are malicious. Especially the exported part.
Application may export sensitive routines. Or may send intents to untrusted app
that may steal information. 
\item Intent filters are described in manifest along with the component. When
intents are sent, they can explicit mention component name, or mention the
filter that an application has to satisfy. Filter categories are action, data
category.
\subsection {Malicious component is receiver}
\begin {itemize} \itemsep -2pt
\item Broadcast receivers can get broadcasts in an unauthroised way. If it can
register as high priority receiver, it can receive intents and not propogate them
resulting in DoS attack.
\item Activity can steal data. Can spoof the interface and perform phisihing, false
response attack, etc. Return malicious response value that changes program state.
\item Service can steal data, phish, spoof. If application provides call backs
can do additional damage.
\item Special intents delegate intent permissions sent with a msg. They can be
used to grant permission to write to Content Providers, or perform actions, which
can be passed down a line. Can do damage to app's datastore.
\end {itemize}
\subsection {Malicious component is sender: Intent spoofing}
\begin {itemize} \itemsep -2pt
\item Broadcast receivers can be tricked to perform senstive tasks by a spoofed
intent.
\item Malicious application can change program state, when sending an intent to
an activity.
\item Malicious application can change program state, when sending an intent to
a service. Can use service's methods carelessly exposed to do wicked stuff. Service
can return sensitive information back to the caller.
\end {itemize}
\end {itemize}
\subsection {Verification}
\begin {itemize} \itemsep -2pt
\item Track intent objects, filters, data, permissions attached and track if sent
inappropriately as an implicit intent at various sinks.
\item Track components in manifests and see if they are not unwittingly exported.
\end {itemize}
\section {CHEX}
\begin {itemize} \itemsep -2pt
\item Since exported components, are a vital part of android, it is important
that they are safe. A component might be unwittingly exported. Declaring intent
filters forces a component to be exported.
\item CHEX looks at exported components, identifies different event handlers that
can be called by android framework. It tracks data flows, dividing the control
flow from entry points to splits.
\item The data flow from sensitive sources to sinks can occur during an asynchronous
call the component serves. For example during one call, a service may get Location
information, store it in a instance variable, and then for a next call send it
over a network. CHEX has to find when more than one flows contribute to leak.
\item The flow of information from sensitive sources, to sensitive sinks is tracked.
In a class, it can flow through instance variables or pair funcions (setters
followed by getters).
\item CHEX permutes splits and performs flow analysis, to see if any execution 
can cause sensitive information to be read from an app, written to an app or both.
\end {itemize}

\end{document}

